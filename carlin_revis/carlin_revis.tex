\documentclass{article}
\usepackage[
bibencoding=auto
,backend=biber
,sorting=ynt
% ,style=verbose-ibid
% ,citetracker
% ,sorting=none
% ,autolang%error: no value specified for autolang.
]{biblatex}
\usepackage{csquotes, datetime2, keyfloat}
\usepackage{amsmath, amssymb, amsthm}
\usepackage{algorithmic}
%\newcounter{res}
%\newtheorem{result}[res]{Result}
%\usepackage{filecontents}
\addbibresource{../rogard.bib}% syntax for version >= 1.2
\usepackage{hyperref}
\ProvideDocumentCommand{\docfillblank}{}{\begin{minipage}[t]{\linewidth}\end{minipage}}
\ProvideDocumentCommand{\origDate}{}{\DTMdate{2007-05-14}}
\ProvideDocumentCommand{\ThisDate}{}{\DTMdate{2022-04-04}}

\begin{document}

\title{Simulation to solve Bayesian sequential clinical trial problem using parametric decision boundaries\footnote{Revisions: \url{https://github.com/erwannr/statistics/commits/main/ratio_nc}}}

\author{Erwann Rogard}
\date{\ThisDate}

\maketitle

\begin{abstract}
We revisit a Bayesian sequential clinical trial problem\cite{carlin1998} by proposing efficient simulation schemes for finding the
optimal parameters of the continuation region (equivalently stopping
rule). This is a re-edition of a note from \DTMdate{2006-10-03}\footnote{\url{https://github.com/erwannr/statistics/tree/main/carlin_revis}}.
\end{abstract}

\section{Introduction}
In \autoref{sec:formul}, we introduce the problem formulation for a simple
model and utility framework, starting from the intuitive (but
practically not recommended) Bellman recursion, and translating it
into a parametric stopping rule (SR) approach, which is the focus of
our study.

Under SR we have to maximize an expected utility over decision
boundaries  that have a parametric representation. Without a closed
form for the expected utility, it has to be approximated from
simulation samples. Stochastic optimization methods are therefore
considered.

In general, for this type of problem, there is no straightforward
``interchange of differentiation and expectation'' formula that
would allow to incorporate derivative information into the algorithm
directly. To address this issue, in Section 3, we present a finite
difference approach to the problem, namely the Kiefer--Wolfowitz
algorithm. The challenge, here, is variance reduction, as the use of
finite difference induces a particular ``importance region''. In
Section 4, we re--express the gradient as a multidimensional
integral restricted to a hypercube region of the space. By treating
the integrand times the indicator for the hypercube as an
un--normalized density, the problem becomes amenable to importance
sampling or MCMC.

In Section 5, we consider more realistic models, implying more
complicated and unknown policy spaces.

In Section 6, we address convergence issues from a theoretical
standpoint, and bridge our work with established sequential trials'
methods/ and or an application

\section{Problem formulation}\label{sec:formul}
Consider a sequential clinical trial extending over a maximum of $n$
periods to assess the effect $\theta$ of a treatment, $T$, in
comparison to the placebo, $P$. Suppose that at period $i$, $0\leq
i\leq n$, we have accumulated data $\mathbf{y}_i=(y_0,\ldots,y_i)$,
with $y_0$ as prior. In this situation, our action space is
$\mathcal{A}=\{T,P,C\}$, where $C$ means continue to $i+1$, which
incurs a sampling cost $c_{i+1}$. Let
$\mathbf{a}_i=\{a_j\}_{j=0}^{i}$ denote the sequence of actions up
to period $i$. For that period, our utility (or negative loss) is
\begin{eqnarray}\label{util}
u(i,\theta,a)&=&\left\{
\begin{array}{ll}
-c_{i,T}(c_P-\theta)^{+}-c_j,& a=T\\
-c_{i,P}(\theta-c_T)^{+}-c_j,&
a=P\\
-\infty 1\{i=n\},& a=C
\end{array}\right.
\end{eqnarray}where the $c_{i,a}$'s denote penalties, and
$[c_P,c_T]$ an indifference zone. The cost of forming the prior,
$c_0$, does not affect the subsequent decisions, so we assume
$c_0=0$. If we know that $a_{i-1}=C$ and $a_i\neq C$, then the
relevant action space at period $i$ is
$\mathcal{A}^{\mathrm{s}}=\{T,P\}$, and for $j>i$,
$\mathcal{A}_{j}=\{\varnothing\}$ and $u(j,\theta,\varnothing)=0$.
The utility up to period $i$ is
$u(\theta,\mathbf{a}_i)=\sum_{j=0}^{i}\beta^{j}u(j,\theta,a_j)$
where $0\leq\beta\leq 1$ is a discount factor. For now we assume
$\beta=1$.

%say why this is a MDP

Our problem is to characterize a policy,
$a:\boldsymbol{\mathcal{Y}}_i\rightarrow\mathcal{A}$, which
maximizes the expected utility. Let
$u(\mathbf{y}_i,a)=\mathrm{E}[u(\theta,\mathbf{a}_{i})|\mathbf{y}_i,a_{i-1}=C,a_{i}\neq
C]$. Within this restricted action space, we have to solve $
\max_{a\in\mathcal{A}^{\mathrm{s}}} u(\mathbf{y}_i,a)$, whose
expected utility and solution we denote $u(\mathbf{y}_i)$ and
$a^{\mathrm{s}}(\mathbf{y}_i)$, respectively.

We will first study the model
\begin{equation}
\label{normal_normal} p(\theta,\mathbf{y}_{n}) =
N(\theta|\mu,\sigma_0)\Pi_{i=1}^{n} N(y_i|\theta,\sigma_i)
\end{equation}where $\mu,\sigma_0,...,\sigma_{n}$ are given
parameters. Under this model, the policy for
$\mathcal{A}^{\mathrm{s}}$ has a simple characterization:
\begin{eqnarray}\label{T_or_P}a^{\mathrm{s}}(\mathbf{y}_i)=\mathrm{T}
\Leftrightarrow \mu(\mathbf{y}_{i})\geq \mu_i\end{eqnarray}where
$\mu_i=\frac{c_{i,T}c_{P}-c_{i,P}c_{T}}{c_{i,T}+c_{i,P}}$ and
$\mu(\mathbf{y}_i)=E[\theta|\mathbf{y}_i]$.

The sequential problem is characterized by the Bellman recursion
(BR):
\begin{eqnarray}\label{BR_terminal}
u_*(\mathbf{y}_{n}) &=& u(\mathbf{y}_{n})\\
\label{BR_cont}
\upsilon(\mathbf{y}_i)&=&\mathrm{E}[u_*(\mathbf{y}_{i+1})|\mathbf{y}_i]\\
\label{BR_opt}
u_*(\mathbf{y}_i)&=&\max(u(\mathbf{y}_i),\upsilon(\mathbf{y}_i))
\end{eqnarray}for $i=i_{\max-1},...,0$, where $\upsilon(\mathbf{y}_i)$ and
$u_*(\mathbf{y}_{i+1})$ are the expected utility, from moving to
$i+1$ (and pursuing an optimal strategy thereafter), and from making
the overall best decision. Clearly,
$u_*(\mathbf{y}_i)=\upsilon(\mathbf{y}_i)\Rightarrow
a(\mathbf{y}_i)=C$, otherwise,
$a(\mathbf{y}_i)=a^{\mathrm{s}}(\mathbf{y}_i)$.

The above problem is solved in two steps: \theoremstyle{definition}
\newtheorem{algorithm}{Algorithm}
\begin{algorithm}[BR]\label{alg:BR}
\hfill\par
\begin{enumerate}\item Fix $m$, and starting from
$\mathbf{y}_0$, apply the recursion $y_{i+1}^{j_1,...,j_{i},j_{i+1}}
\stackrel{\mathrm{iid}}{\sim}
p(y_{i+1}|\mathbf{y}_i^{j_1,...,j_{i}})$, $j_{i+1}=1,...,m$, $0\leq
i<n$, resulting in a tree structure rooted at $\mathbf{y}_0$ and
$m^{n}$ terminal nodes
\item At all the nodes at $i=n$ compute (\ref{BR_terminal}) and for every other node,
in the order $i=n-1,...,0$ compute (\ref{BR_cont}--\ref{BR_opt})with
(\ref{BR_cont}) approximated as follows:
\begin{align}\label{BR_est_upsilon}
\hat\upsilon(\mathbf{y}_i^{j_1,...,j_i})=\frac{1}{m} \sum_{j=1}^m
\hat u_*(\mathbf{y}_{i+1}^{j_1,...,j_i,j})
\end{align}
\end{enumerate}
\end{algorithm}

The above method has the following characteristic:
\begin{enumerate}
\item The simulated tree implies a computational demand exponential in $n$
\item The relevant distribution, at each node, is the predictive
density i.e. $p(y_{i+1}|\mathbf{y}_{i})$.
\item A trivial optimization problem, (\ref{BR_opt}), is solved at each node of the
simulated tree
\item The method is independent of the problem's structure (model and loss)
\item The bias of (\ref{BR_est_upsilon}) is positive for finite $m$,
and zero in the limit as $m\rightarrow \infty$
\end{enumerate}In the iid case,
$p(y_{i+1}|\mathbf{y}_{i})=\mathrm{E}[p(y_{i+1}|\theta)|\mathbf{y}_{i}]$
which in general is not easy to sample.

The first characteristic means that algorithm (\ref{alg:BR}) is
impractical beyond 2 or 3 periods. However, the memory requirement
can be made linear in $n$ by depth first processing:
%\begin{algorithm}[BR--DF]\label{alg:BR--DF}
%\hfill\par
%\begin{enumerate}
%\item Define \\$\upsilon(\mathbf{y},i,n,m)\{\\u\leftarrow u(\mathbf{y}_i);\\
%\mathrm{for}(j=1\quad \mathrm{to}\quad m)\{\\\mathrm{generate}\quad
%\mathbf{y}^j\sim p(\mathbf{y}^j|\mathbf{y}_i)\}\\\}$
%\end{enumerate}
%\end{algorithm}

\subsection{independent paths alternative}
To remedy the first limitation of algorithm \ref{alg:BR} we would
like replace the simulated tree in the first tree by independent
paths and correct with importance sampling. Consider the formulation
of (\ref{BR_cont}) as
\begin{eqnarray}\label{BR_cont_alt_1}
\mathrm{E}[u_{*}(\mathbf{y}_{i+1})|\mathbf{y}_{i}]
&=\mathrm{E}_*[u_{*}(\mathbf{y}_{i+1}^*)\frac{p(\mathbf{y}_{i+1}^*|\mathbf{y}_{i}^*=\mathbf{y}_{i})}{p(\mathbf{y}_{i+1}^*)}]\\\label{BR_cont_alt_2}&=
\mathrm{E}_*[u_{*}(\mathbf{y}_{i+1}^*)\frac{p(\mathbf{y}_{i+1}^*|\mathbf{y}_i^*=\mathbf{y}_i)}{\mathrm{E}_{**}[p(\mathbf{y}_{i+1}^*|\mathbf{y}_{i}^{**}=\mathbf{y}_{i}^*)]}]
\end{eqnarray}where $\mathbf{y}_i^*$ indicates a variable
independent of $\mathbf{y}_i$ but with the same distribution. In
order for the above to make sense, we have to take $\mathbf{y}_i$ as
meaning the sufficient statistic for $\theta$ of the data
accumulated up to period $i$, and assume\footnote{see Glasserman,
2000, Section 8.5.1 for all conditions}
\begin{align}\label{mesh_cond_1}p(\mathbf{y}_{i+1}|\mathbf{y}_{i-1},...,\mathbf{y}_{1})=p(\mathbf{y}_{i+1}|\mathbf{y}_{i-1}).\end{align}
The new procedure is:
\begin{algorithm}[BR--IS]\label{alg:BR_IS}
\hfill\par
\begin{enumerate}
\item Simulate $m$ iid paths with transition
$p(\mathbf{y}_{i+1}|\mathbf{y}_{i}), \quad i=0,...,n-1$
\item At all the nodes at $i=n$ compute
(\ref{BR_terminal}) and for every other node, in the order
$i=n-1,...,0$ compute (\ref{BR_cont}--\ref{BR_opt}) with the
(\ref{BR_cont}) approximated as follows:
\begin{align}\label{BR_IS_est_upsilon}
\hat\upsilon(\mathbf{y}_i^{j})&=\frac{1}{m} \sum_{k=1}^m
w(\mathbf{y}_{i+1}^k|\mathbf{y}_{i}^j)\hat u_*(\mathbf{y}_{i+1}^{k})
\end{align}and the importance weights computed as either of the two
forms\footnote{see the discussion of Glasserman, 2000, Section 8.5.2
for a discussion of the superiority of the second form}:
\begin{align}
w(\mathbf{y}_{i+1}^k|\mathbf{y}_{i}^j)&=\frac{p(\mathbf{y}_{i+1}^k|\mathbf{y}_{i}^j)}{p(\mathbf{y}_{i+1}^k)}\\
w(\mathbf{y}_{i+1}^k|\mathbf{y}_{i}^j)&=\frac{p(\mathbf{y}_{i+1}^k|\mathbf{y}_{i}^j)}{\frac{1}{m}\sum_{l=1}^m
p(\mathbf{y}_{i+1}^k|\mathbf{y}_{i}^l)}
\end{align}
\end{enumerate}
\end{algorithm}

\subsection{stopping rule formulation}
We define
\begin{align}\label{tau_*}
\tau_*\triangleq\min\{i: u(\mathbf{y}_i)>\upsilon(\mathbf{y}_i)\}
\end{align}and $u(\theta,\mathbf{y}_i)\triangleq
u(i,\theta,a(\mathbf{y}_i))$. In view of (\ref{BR_terminal}),
(\ref{BR_cont}) and (\ref{BR_opt}),
$v(\mathbf{y}_i)=\mathrm{E}[u(\theta,\mathbf{y}_{\tau_*})|\mathbf{y}_i]$,
and in particular,
\begin{eqnarray}\label{opt_cont}
\upsilon\triangleq\upsilon(\mathbf{y}_0)=\mathrm{E}[u(\theta,\mathbf{y}_{\tau_*})]
\end{eqnarray}where $\mathrm{E}[.]$ is understood as conditioning on
$\mathbf{y}_0$ to alleviate subsequent notation. With the above
stopping rule, a given path $\mathbf{y}$ is either stopped before
$i$, at $i$ or beyond $i$. This defines a partition
$\boldsymbol{\mathcal{Y}}_i=\mathbf{C}_{i-1}^{\mathrm{c}}\cup\mathbf{S}_i
\cup \mathbf{C}_i$. The relationships between these sets are
$\mathbf{C}_i=\mathbf{C}_{i-1}\cup C_{i}$, for $i=1,...,n-1$,
$\mathbf{C}_0=C_{0}$ and $\mathbf{S}_i=\mathbf{C}_{i-1}\cap
C_{i-1}^{\mathrm{c}}$ where
%and partition the space $\boldsymbol{\mathcal{Y}}_i$ as:
%\begin{align}
%\mathbf{T}_i & =\{\mathbf{y}_i: a(\mathbf{y}_j)=C,j<i\quad \mathrm{and}\quad a(\mathbf{y}_i)=T\}\\
%\mathbf{P}_i & =\{\mathbf{y}_i: a(\mathbf{y}_j)=C,j<i\quad \mathrm{and}\quad a(\mathbf{y}_i)=P\}\\
%\mathbf{C}_i & =\{\mathbf{y}_i: a(\mathbf{y}_j)=C,j\leq i\}\\
%\mathbf{C}_{i-1}^{\mathrm{c}} & =\{\mathbf{y}_i: a(\mathbf{y}_j)\neq
%C,j<i\}
%\end{align}Using the above, recursively for $j=i-1,...,1$,
%we get the finer partition:
%\begin{align}
%\boldsymbol{\mathcal{Y}}_i=\mathbf{T}_1 \cup \mathbf{P}_1...\cup
%\mathbf{T}_i \cup \mathbf{P}_i \cup \mathbf{C}_i
%\end{align}In view of (\ref{BR_terminal}), (\ref{BR_cont}) and (\ref{BR_opt}),
\begin{align}\label{C_i} C_i &=\{\mathbf{y}_{i}:
\upsilon(\mathbf{y}_i)>u(\mathbf{y}_i)\}, \quad
i=0,...,n-1\\\label{C_0} C_{n}&=\{\emptyset\}
\end{align}
We can estimate the stopping rule as follows. Suppose have generated
$j=1,...,m$ paths and computed
$(u(\mathbf{y}_i^j),\hat\upsilon(\mathbf{y}_i^j))$ by BR--IS, for
each node $(i,j)$. For any given $\mathbf{y}_i^*$, independent of
the previous draws, we can estimate $\upsilon(\mathbf{y}_i^*)$ by
plugging $\mathbf{y}_i^*$ in place of $\mathbf{y}_i^j$ into
(\ref{BR_IS_est_upsilon}). By plugging the estimate,
$\hat\upsilon(\mathbf{y}_i^*)$, into (\ref{tau_*}) results in an
estimate $\hat\tau$. By plugging $\hat\tau$ in place of $\tau_*$ in
(\ref{opt_cont}) we obtain:
\begin{align}\label{BR_IS_est_upsilon_neg}
\upsilon(\hat \tau)&\triangleq
\mathrm{E}[u(\theta,\mathbf{y}_{\hat\tau})]\\&< \upsilon
\end{align}As we recall the BR estimator (and therefore the
BR--IS) of $\upsilon$ a positive bias, whereas
(\ref{BR_IS_est_upsilon_neg}) has a negative bias. By combining the
two estimators, we can therefore obtain an interval which contains
the true value $\upsilon$ for a given confidence level.

\section{Parametric decision boundaries}\label{sec:param}

The second way to estimate the stopping rule is to model it. Let
$C=(C_1,...,C_{n-1})$ and suppose we postulate $\mathbf{C}:\Gamma
\rightarrow\boldsymbol{\mathcal{Y}}^{n-1}$. As we saw, the
continuation regions determine the stopping rule, which in turn
determines a continuation value:\begin{align}
\tau(\gamma)&=\min\{i: \mathbf{y}_i \notin C_i(\gamma)\}\\
\label{upsilon_par_sr} \upsilon(\gamma)&=
\mathrm{E}[u(\theta,\mathbf{y}_{\tau(\gamma)})]
\end{align}Our objective is now the maximization of
(\ref{upsilon_par_sr}). Clearly, we should expect
 \begin{align}\label{sup_upsilon_par_sr}
\sup_{\gamma\in\Gamma} \upsilon(\gamma)\leq \upsilon
\end{align}
Recalling the characteristics of BR at the end of the previous
Section, let us now contrast them with those of SR:
\begin{enumerate}
\item Estimating the expectation in (\ref{upsilon_par_sr}) requires parallel
paths
\item The relevant distribution is $p(\theta,\mathbf{y}_{\tau})$
\item We have to optimize over all entries of $\gamma$ simultaneously.
% make this more explicit
\item The structure of the optimization problem depends on the
distribution and utility under consideration
\end{enumerate}Assuming iid data
$p(\theta,\mathbf{y}_\tau)=p(\theta)\Pi_{j=1}^{\tau}p(y_j)$, so it
is essential that $p(\theta)$ be ``proper'' for sampling purposes.
Practically, this often means restricting the search to conjugate
priors, or otherwise incorporating some actual data into the prior
(see Gelman et al., Section 4.3).

Under (\ref{util}) and (\ref{normal_normal}), Carlin et al. (1998)
have derived the true model, i.e. that which achieves equality for
(\ref{sup_upsilon_par_sr}). It is characterized as follows:
\begin{align}\label{gamma}
\gamma&=(\gamma_1,...,\gamma_{2(n-1)})\\
\label{C_i_param} C_{i}(\gamma)&= \{\mathbf{y}_i:
\gamma_{i}^{-}\leq\mu(\mathbf{y}_{i})\leq \gamma_{i}^{+}\}, \quad
1\leq i\leq n-1\\\label{Gamma} \Gamma&=\{\gamma: \gamma_k \in
\mathbb{R}, \gamma_i^{-}\leq \mu_i \leq\gamma_i^{+}\}
\end{align} where $\gamma_{i}^{-}\triangleq\gamma_{2(i-1)+1}$,
$\gamma_{i}^{+}\triangleq \gamma_{2i}$. We define $i(k)=\lfloor k+1
\rfloor/2$.

From (\ref{gamma}--\ref{Gamma}), and recalling (\ref{T_or_P}), we
see that $\mu(\mathbf{y}_{i})$ summarizes the observable state
variables. Moreover, $\mu(\mathbf{y}_i)=\mathbf{s}_i
\mu(\mathbf{y}_{i-1})+(1-\mathbf{s}_i)y_{i+1}$ with
$\mu(\mathbf{y}_0)=\mu$ and $y_{i+1}=\theta+\epsilon_{i+1}$ so that
conditional on $\theta$, $\{\mu(\mathbf{y}_i)\}_{i=0}^{n}$ evolves
as a Markov chain.

\subsection{The optimization problem}
The algorithms that we will consider to solve
(\ref{sup_upsilon_par_sr}) start from an initial guess
$\gamma^{(0)}$, and update it in an iterative fashion so as maximize
a local approximation to (\ref{upsilon_par_sr}):
\begin{eqnarray}\label{KW}
\gamma^{(b+1)}&=&\mathrm{Proj}_{\Gamma}(\gamma^{(b)}-\alpha^{(b)}
{H^{(b)}}^{-1}\widehat{\nabla}^{(b)} \upsilon)
%\\
%\label{SFD} \nabla_{\mathrm{\scriptscriptstyle FD}}^{(b)}
%v_{\mathrm{sim}} &=&\{
%\frac{1}{h_{k}^{(b)}}(v_{\mathrm{sim}}(\gamma^{(b)}+e_{k}h_{k}^{(b)})-v_{\mathrm{sim}}(\gamma^{(b)}))\}_{k=1}^{\mathrm{dim}(\Gamma)}
\end{eqnarray}where
%$v_{\mathrm{sim}}(.)$ replaces $\mathrm{E}[.]$
%in (\ref{upsilon_par_sr}) by the average over $n$ simulation draws (an
%independent sample for each $b$).
$\alpha^{(b)}$, $H^{(b)}$, are the step size, and scaling matrix
(ideally the the hessian). We will consider different ways to
compute $\widehat{\nabla} \upsilon^{(b)} \approx \nabla
\upsilon(\gamma^{(b)})$ using averages over simulation samples. The
particular distribution that we choose for $\widehat{\nabla}
\upsilon$ at iteration $(b)$, say $d$, induces a distribution, say
$d$ for $\gamma^{(b+1)}$. Our objective, therefore, can be expressed
as
\begin{equation}
\min_d \mathrm{E}_d[||\gamma_*^{(b+1)}-\gamma^{(b+1)}||]
\end{equation}where $\gamma_*^{(b+1)}$ is optimal in some sense, for
example the solution of (\ref{KW}) assuming perfect knowledge of
$\nabla_2 \upsilon(.)$ and $\nabla \upsilon(.)$ \footnote{In this
case, $\Delta_q=\mathrm{E}_q[||\nabla_2^{-1}
\upsilon(\gamma^{(b)})\nabla
\upsilon(\gamma^{(b)})-{H^{(b)}}^{-1}\widehat\nabla^{(b)}\upsilon_{\mathrm{sim}}||]$}.
This expression is intricate. A reasonable intermediary objective,
however, is the minimization of
\begin{equation}\label{gradient_distance}
\Delta_d^{(b)} = \mathrm{E}_d[||\nabla
\upsilon(\gamma^{(b)})-\widehat\nabla^{(b)} \upsilon||]
\end{equation}(TODO: what norm is appropriate? $||.||_{H^{-1}}$?)

Consider the following decomposition:
\begin{align}\label{upsilon_par_sr_i}
\upsilon(\boldsymbol{\gamma}_i)&=\mathrm{E}[u(\theta,\mathbf{y}_i)1\{\tau(\gamma)=i\}]\\
\label{upsilon_par_sr_decomp} \upsilon(\gamma)
&=\upsilon(\boldsymbol{\gamma}_1)+...+\upsilon(\boldsymbol{\gamma}_{n-1})
\end{align}where
$\boldsymbol{\gamma}_i=\{\gamma_k\}_{k=1}^{2i}$. Under
(\ref{upsilon_par_sr}), the quantity (\ref{upsilon_par_sr_i}) has
the representation
\begin{equation}\label{upsilon_par_sr_i_integral}
\upsilon(\boldsymbol{\gamma}_i)=-\mathbf{c}_{i-1}+\int_{\mathbb{R}}\idotsint\limits_{\mathbf{R}_{i-1}}\Bigl[\int_{-\infty}^{\gamma_i^{-}}u(i,\theta,T)p(\theta,\mathbf{y}_{i})
d
y_i+\int_{\gamma_i^{+}}^{\infty}u(i,\theta,P)p(\theta,\mathbf{y}_{i})d
y_i\Bigr]d \mathbf{y}_{i-1}  d \theta
\end{equation}with $\gamma_{n}^{\pm}=\mu_{n}$ and $\mathbf{c}_i=\sum_{j=1}^{i}
c_j$. Let
\begin{equation}\Upsilon(\boldsymbol{\gamma}_i)=\boldsymbol{1}\{\mathbf{y}_{i-1}\in
\mathbf{R}_{i-1}, \}(\boldsymbol{1}\{\mu(\mathbf{y}_i)\leq
\gamma_{i}^{-}\}u(i,\theta,T)+\boldsymbol{1}\{\mu(\mathbf{y}_i)\geq
\gamma_{i}^{+}\}u(i,\theta,P))p(\theta,\mathbf{y}_{i})\end{equation}we
have
$\upsilon(\boldsymbol{\gamma}_i)=\mathrm{E}[\Upsilon(\boldsymbol{\gamma}_i)]$.
However, the presence of indicator functions implies $\nabla
\upsilon(\boldsymbol{\gamma}_i)\neq
\mathrm{E}[\nabla\Upsilon(\boldsymbol{\gamma}_i)]$, which precludes
a pathwise simulation approximation to the derivative. To resolve
this difficulty, we will consider two methods.

The first is the finite difference method. This involves only
evaluating (\ref{upsilon_par_sr_decomp}) at $\boldsymbol{\gamma}_i$
and $\boldsymbol{\gamma}_i+\Delta$ for some small $\Delta$. As
already noted, simulating from $p(\theta,\mathbf{y}_i)$ should not
present major difficulties in the iid case. However, we are not
interested in simulating $p(.)$, rather a truncated version of it.
Importance sampling solutions will be proposed to improve
efficiency.

The second method is to take partial derivatives of
(\ref{upsilon_par_sr_i_integral}) with respect to $\gamma_j$. The
resulting expression is no longer an expectation with respect to a
well defined distribution. However, we can treat this expression as
a un un--normalized density, and since it can be evaluated, the
problem amenable to importance sampling or Markov chain simulation.

\section{Method 1}\label{sec:meth:i}
\subsection{Procedure}
We now take $\widehat\nabla \upsilon$ to be a simulation
approximation to finite difference gradient,
$\widehat{\nabla}_{\mathrm{\scriptscriptstyle FD}} \upsilon$. In
this case, the recursion (\ref{KW}) is due to Kiefer--Wolfowitz. Let
us start from the classic finite difference expression for
$\nabla_{\mathrm{\scriptscriptstyle FD}} \upsilon$. Fix $(b)$, so
that $\tau=\tau(\gamma^{(b)})$, and define $\tau_k=\tau(\gamma+e_k
h_k)$. Throughout, we will assume a sequential importance form for
the finite difference gradient:
\begin{equation}\label{nabla_FD_IS}
\nabla_{\mathrm{\scriptscriptstyle FD}}v=\mathrm{E}_q[(w\times
\nabla_{\mathrm{\scriptscriptstyle FD}}
u)(\theta,\mathbf{y}_{\tau})]
\end{equation}
The $k^{\mathrm{th}}$ element of the integrand is
\begin{equation}\label{FD_u}
(w\times \nabla_\mathrm{\scriptscriptstyle FD}
u(\theta,\mathbf{y}_{\tau}))_k=\frac{(w\times
u)(\theta,\mathbf{y}_{\tau_k})-(w\times
u)(\theta,\mathbf{y}_{\tau})}{h_k}
\end{equation}where, $q(.)$ is a distribution of our
choice within the class $q\gg p$,
$w(\theta,\mathbf{y}_{\tau})=p(\theta,\mathbf{y}_{\tau})/q(\theta,\mathbf{y}_{\tau})$,
and $\times$ is the product operator i.e. $(f\times g)(.)=f(.)g(.)$.
The resulting expression for (\ref{nabla_FD_IS}) will equal $\nabla
\upsilon$ plus a residual term that converges to zero as
$||h||\rightarrow 0$ where $h=(h_1,\ldots,h_{2(n-1)})$.

The procedure to estimate (\ref{nabla_FD_IS}) is to sample $n$ iid
paths from
$q(\theta,\mathbf{y}_{\max(\tau,\tau_1,...,\tau_{2(n-1)})})$ and
evaluate the sample average
$\widehat{\nabla}_{\mathrm{\scriptscriptstyle FD}}
\upsilon=\widehat{\mathrm{E}}_q[(w\times
\nabla_{\mathrm{\scriptscriptstyle FD}}
u)(\theta,\mathbf{y}_{\tau})]$. Each term of difference in
(\ref{FD_u}) is evaluated with the \emph{same} path, which reduces
variance, provided there is positive correlation between the two
terms. For each iteration $(b)$, a new set of $n$ paths is
simulated, which is indicated in (\ref{KW}) by the superscript$(b)$
of $\widehat{\nabla}_{\mathrm{\scriptscriptstyle FD}}^{(b)}
\upsilon$. This is part of the conditions for convergence and stems
from the fact that averaging of the $\gamma^{(b)}$'s is already
present, as in implicit in (\ref{KW}). If it is felt that the
behavior of the algorithm is erratic, a larger value for $n$ or a
smaller value for $\alpha^{(0)}$ is recommended.
%It is not recommended to smooth the $\widehat{\nabla}^{(b)} \upsilon$'s across
%past iterations (this would reduce variance and introduce additional
%bias) or reuse past samples, because .

We can weaken the above formulation to
$w(\theta,\mathbf{y}_{\tau})\propto
p(\theta,\mathbf{y}_{\tau})/q(\theta,\mathbf{y}_{\tau})$, with an
arbitrary normalizing constant, which imposes the requirement that
the sample $w$'s be normalized to one. This may be a matter of
design (to make the sample average equivariant, see Hestenberg,
1995) or necessity, in case either of $q(.)$ or $p(.)$'s normalizing
constant is not known. Unbiasedness of
$\widehat{\nabla}_{\mathrm{\scriptscriptstyle FD}} \upsilon$,
relative to $\nabla_{\mathrm{\scriptscriptstyle FD}} \upsilon$,is
only preserved in the limit as $n\rightarrow \infty$. If we
explicitly need to distinguish this method from the standard one, we
will write $\tilde w$ and $\widetilde{\mathrm{E}}_q[.]$.

%Moreover, this method induces dependence between the paths, so that
%the sample variance of the $(\tilde w \times
%u)(\theta,\mathbf{y}_{\tau})$'s is a biased estimate of
%$\mathrm{Var}_q[(\tilde w \times u)(\theta,\mathbf{y}_{\tau})]$; a
%bootstrap estimate might be preferable.

In view of (\ref{gradient_distance}), $\Delta_d \triangleq
\Delta_{q,h}$ defined as
\begin{align}
\Delta_{q,h} &=\mathrm{E}_q[||\nabla\upsilon-\widehat\nabla
_{\mathrm{\scriptscriptstyle FD}}^h\upsilon||]\\
&=\mathrm{E}_q[||\nabla\upsilon-\nabla _{\mathrm{\scriptscriptstyle
FD}}^h\upsilon||+||\widehat\nabla _{\mathrm{\scriptscriptstyle
FD}}^h\upsilon-\nabla _{\mathrm{\scriptscriptstyle FD}}^h\upsilon||]
\end{align}where the finite difference increment $h$ is now shown
explicitly in superscript. One problem with minimizing this
expression is the interaction between $q(.)$ and $h$. A first step
towards simplification is to fix $h$, thereby fixing
$\textsl{bias}_{}(h)=||\nabla_\mathrm{\scriptscriptstyle
FD}^h\upsilon-\nabla\upsilon||$, and focusing on $q(.)$ to reduce
some variance measure (TODO: define it). This approach has practical
interest: often we have a prior belief (say from initial trials)
about $d_k^{(0)}=||\gamma_k^{(0)}-\gamma_{k}^*||$, and it seems
reasonable to set $h\leq \epsilon d_k^{(0)}$, say $h_k^{\max}$. In
our own experimentation, for $q(.)=p(.)$, the value of $h$ that
minimizes $||\nabla_\mathrm{\scriptscriptstyle
FD}\upsilon-\nabla\upsilon||$ can be greater than $d_k^{(0)}$ than
itself, if the curvature of $\upsilon(\gamma^{(0)})$ is small, which
underlines the need to fix an upper bound. Our procedure for
initializing the algorithm, therefore, \begin{enumerate}\item decide
on a appropriate value for $h_{\max}$ \item solve $\min_q
\Delta_{q,h_{\max}}$, yielding $q_*$ \item fine tune the finite
difference increment, $\min_{h\leq
h^{\max}}\Delta_{q_*,h}$.\end{enumerate} It may be possible to
perform $\min_q \Delta_{q,h^{(b)}}$ throughout the course of the
algorithm, where $\{h^{(b)}\}$ is an appropriately diminishing
sequence but in this case a proper convergence analysis is required.

The discussion that follows is also valid for variants of the finite
difference method, with a few adjustments. In particular, if
$\textsl{dim}(\Gamma)$ is large, it may be worthwhile perturbing all
elements of $\gamma$ as opposed to each separately. Let
$\tau_\epsilon=\tau(\gamma+h \epsilon)$,and
\begin{align}\label{FD_u_SP}
(\nabla_\mathrm{\scriptscriptstyle FD}
u(\theta,\mathbf{y}_{\tau}))_k
=\frac{u(\theta,\mathbf{y}_{\tau'})-u(\theta,\mathbf{y}_{\tau})}{h
\epsilon_k}
\end{align}where $\epsilon=\{\epsilon_k\}_{k=1}^{2(n-1)}$ is a random
perturbation vector. This finite difference alternative is termed
``simultaneous perturbation'' (SP). Compared with the previous
formulation, only two evaluations of $u(.)$ are needed, a gain of to
$2(n-1)-1$ evaluations. Under reasonable conditions, this gain more
than offsets the loss in efficiency. Among these conditions,
$E[|1/\epsilon_k|]\leq \infty$, which is verified for independent
$\epsilon_k$'s distributed as $\pm 1$ Bernoulli's. The search for an
optimal $h$ can be derived from that of the standard finite
difference method (see Section tuning the gains parameters)

\subsection{Importance region}
The quantity (\ref{FD_u}) has value $0$ outside of the set
\begin{align}
\label{region_A}
\mathbf{A}_k &= \{\mathbf{y}_{i(k)}: \tau \neq \tau_k\}=\mathbf{R}_{i(k)-1}\cap A_k\\
A_k &=\{\mathbf{y}_{i(k)}:\gamma_k\leq
\mu(\mathbf{y}_{i(k)})\leq\gamma_k+e_k h_k\}\end{align}where
$i(k)=\lfloor k+1\rfloor/2$.
%\begin{align}\label{p_cond_A}
%p_k(\theta,\mathbf{y}_{n})&=p(\theta,\mathbf{y}_{n}|\mathbf{A}_k)\\
%&= \mathbf{1}\{\mathbf{y}_{i(k)}\in \mathbf{A}_k\}
%p(\theta,\mathbf{y}_{i(k)})/p(\mathbf{A}_k)\\\nonumber &=
%p(\mathbf{y}_{i(k)}|\mathbf{A}_k)p(\theta|\mathbf{y}_{i(k)})p(\mathbf{y}_{i>i(k)}|\theta)
%\end{align}
From a computation standpoint, if we simulate along paths in a
stepwise fashion, we may record $0$ for (\ref{FD_u}) for each $k$ as
soon as we observe $\tau<i(k)$.

From an efficiency perspective, the optimal sampling density is
$q_*(.)\propto p(.) \times |(\nabla_{\mathrm{\scriptscriptstyle FD}}
u)_k(.)|$. One way to get closer to this goal is to take
$q(\theta,\mathbf{y}_{i(k)})\approx
p(\theta,\mathbf{y}_{i(k)}|\mathbf{A}_k)$, so that
$w(\theta,\mathbf{y}_{i(k)})\approx p(\mathbf{A}_k)$. This objective
is all the more critical that $\lim_{b\rightarrow\infty} A_k^{(b)} =
\{\mathbf{y}_{i(k)}:\mu(\mathbf{y}_{i(k)})=\gamma_k\}$ implying
$\lim_{b\rightarrow\infty}p(\mathbf{A}_k^{(b)})=0$ i.e.
$\mathbf{A}_k$ is asymptotically a rare event. Note also, that
$\lim_{k\rightarrow\infty}p(\mathbf{A}_k^{(b)})=0$ (curse of
dimensionality TODO: verify).

If the weights are normalized, $\tilde w$, which is common in
Bayesian analysis, the optimal density is $q_*(.)\propto p(.)
|(\nabla_{\mathrm{\scriptscriptstyle FD}}
u)_k(.)-(\nabla_{\mathrm{\scriptscriptstyle FD}} \upsilon)_k|$. In
this case, we need to sample from \emph{both} $\mathbf{A}_k$ and its
complement. Sampling within $\mathbf{A}_k$, however, remains just as
critical and difficult.
%compute the influence function

Based on the preceding remarks, one direction is stratification,
i.e. controlling the number of paths that fall in $\mathbf{A}_k$ and
those falling outside (if any are needed)so that across strata
variance --which in this case is large-- is neutralized.
Specifically, under (\ref{normal_normal}), conditional on $\theta$,
$\mathbf{x}_i\triangleq\mu(\mathbf{y}_{i})=\mathbf{s}_{i}\mu(\mathbf{y}_{i-1})+(1-\mathbf{s}_{i})\theta+(1-\mathbf{s}_{i})\epsilon_i$
so there exists a vector $\mu_i$ such that
$\mu(\mathbf{y}_{i})=\mu_i^{\mathrm{T}}\mathbf{\epsilon}_i$. Fix $k$
and let $\tilde\gamma_{j}^{\pm}=\gamma_{j}^{\pm}$ for  $j<i(k)$,
$\tilde\gamma_{i(k)}^{-}=\gamma_k$ and
$\tilde\gamma_{i(k)}^{+}=\gamma_k+e_k h_k$. Then
$\mathbf{A}_k=\{\mathbf{y}_{i(k)}:
\tilde\gamma_{j}^{-}\leq\mu(\mathbf{y}_j)\leq\tilde\gamma_{j}^{+},
j=1,...,i(k)\}$ which translates into a hypercube for
$\mathbf{x}_i\sim N(\theta\boldsymbol{1}_i ,\Sigma)$. Please read
the relevant section of the appendix. We should expect high
correlation between the $x_i$'s and therefore sampling from
truncated univariate normals will be inefficient. The alternative
method is therefore highly relevant here.

So far we have discussed a single $k$ in isolation, whereas the
search for $q(.)$ should take their joint contribution into account
(Veach and Guibas, 1995). In particular, the $\mathbf{A}_k$'s
overlap in a non trivial way\footnote{$\mathbf{A}_{2(i-1)+1}$ and
$\mathbf{A}_{2(i-1)+2}$ have $\mathbf{R}_{i-1}$ in common and
$A_{k}$ and $A_{k+2}$ have $\mathrm{R}_{i(k)-1}$ in common.},
leading to a complicated resource allocation problem. At the very
least, however, we should expect stratification to work well when
$\gamma$ is altered only in one direction at each step $(b)$.

A second approach is to take $q(.)$ and $p(.)$ identical, except,
for the $\theta$ portion of $(\theta,\mathbf{y}_{\tau})$, leading to
a simple formulation for the importance ratio in the iid case:
$w(\theta,\mathbf{y}_\tau)\propto p(\theta)/q(\theta)$. Moreover,
there are no restrictions of the kind mentioned for stratification.
A convenient approach, here, is to define a family of sampling
distributions indexed by $\lambda$: $\{q_{\lambda}(.),\lambda\in
\Lambda\}$ and select $\lambda$ to minimize some variance criterion.
In general $p(.)$ should be incorporated into $q(.)$ as a defensive
sampling measure (Owen and Zhou, 1998) and also to ``learn'' the
optimal $\lambda$. Please read the relevant section in the appendix
with $q_*(.)\propto p(.) |(\nabla_{\mathrm{\scriptscriptstyle FD}}
u)_k(.)-c)_k|$ where $c$ is $0$ and
$\nabla_{\mathrm{\scriptscriptstyle FD}} \upsilon$, if the weights
$w$ and $\tilde w$ are used, respectively. In view of our previous
discussion, we would like the minimization step to incorporate a
constraint of the form $q_{\lambda}(\mathbf{A}_k)>\epsilon$.

We should note that $\{\nabla \upsilon\}_{\mathrm{odd}}=\{(\nabla
\upsilon)_{2(i-1)+1}\}_{i=1}^{n-1}$ and $\{\nabla
\upsilon\}_{\mathrm{even}}=\{(\nabla
\upsilon)_{2(i-1)+2}\}_{i=1}^{n-1}$ form two homogeneous groups in
the following way: the expressions for each involve $a=T$ and $a=P$,
respectively. Provided the cost structure, for either of
$a\in\{T,P\}$, changes smoothly enough from $i$ to $i+1$, we may
expect some positive correlation between the estimates of members of
each group. However, a key additional assumption is required: that
the members of $\gamma_{\mathrm{odd}}$ and $\gamma_{\mathrm{even}}$
are also smooth along $i=1,...,{n}$. We should therefore impose this
requirement to set the starting value $\gamma^{(0)}$. Since we can
expect that this condition will be verified for
$\gamma^{\infty}=\gamma_*$, provided the algorithm converges, some
level of smoothness should be enforced for each $(b)$. Moreover,
neighboring members of each group should be close in expectation.
Several implications follow:
\begin{enumerate}
\item This suggests defining $q_\lambda(.)$ as mixture of
distributions, one ``taking care'' of each group.
\item For a given computational budget, particularly if
$\beta<1$, it may be better to favor paths that stop early, as the
later periods contribute a diminishing portion to $\upsilon$.
Accordingly, we would set $q_{\lambda}(.)$ to meet that objective.
TODO: define the resource allocation problem.
\item Smoothing
might be appropriate across neighboring $k$'s, within each of
$\{\nabla \upsilon\}_{\mathrm{odd}}$ and $\{\nabla
\upsilon\}_{\mathrm{even}}$ or alternatively
$\{\gamma\}_{\mathrm{odd}}$ and $\{\gamma\}_{\mathrm{even}}$. This
form of smoothing should ne confused with smoothing across $(b)$'s,
which --as already mentioned-- is not valid for estimating $\nabla
\upsilon$ as part of (\ref{KW}).
\end{enumerate}

%$\mathbf{s}_{i+1}=\boldsymbol{\sigma}_i^{-2}\boldsymbol{\sigma}_{i+1}^2$
%and $\boldsymbol{\sigma}_i^2=(\sum_{j=0}^n \sigma_j^2)^{-1}$.

%Here we suggest using a different sampling strategy for each $A_k$.
%Convenient approximations to (\ref{p_cond_A}) are
%\begin{eqnarray} \label{p_cond_A_approx}
%q_{k}(\theta,\mathbf{y}_{2})&=&
%\mathrm{unif}_{\mathrm{A}_{k}}(y_1)p(\theta|y_1)p(y_2|\theta), \quad \mathrm{or}\\
%q_{k}(\theta,\mathbf{y}_{2})&=&
%p(\theta)\mathrm{unif}_{\mathrm{A}_{k}}(y_1)p(y_2|\theta)
%\end{eqnarray}with $w_{k}(\theta,\mathbf{y}_{1})=w_{k}(\theta,\mathbf{y}_{2})=\frac{p(y_1|
%\mathrm{A}_{k})}{\mathrm{unif}_{\mathrm{A}_{k}}(y_1)}$ and
%$w_{k}(\theta,\mathbf{y}_{1})=w_{k}(\theta,\mathbf{y}_{2})=\frac{p(y_1|\theta)/p(
%A_{k})}{\mathrm{unif}_{\mathrm{A}_{k}}(y_1)}$, respectively.
\subsection{Tuning the gains sequences}
Fix $q$ and $k$. The objective, is to balance the bias and variance
of $(\widehat{\nabla}_{\mathrm{\scriptscriptstyle FD}} \upsilon)_k$,
as part of the system (\ref{KW}). It is a classical result that the
system
\begin{eqnarray}\label{alpha_seq}
\alpha^{(b)}&=& \alpha_0/(b+1+B)^{\alpha_{\mathrm{
\scriptscriptstyle rate}}}\\\label{h_seq} h_k^{(b)}&=&
h_k^{0}/(b+1)^{h_{\mathrm{\scriptscriptstyle rate}}}
\end{eqnarray}with the choice of parameters $\alpha_{\mathrm{\scriptscriptstyle rate}}=1$ and
$h_{\mathrm{\scriptscriptstyle rate}}=1/6$, is asymptotically (in
$b$) optimal, and yield a convergence rate for (\ref{KW}) of
$(\alpha_{\mathrm{\scriptscriptstyle rate}}-2
h_{\mathrm{\scriptscriptstyle rate}})/2=1/3$. As the iteration
proceeds, taking smaller steps $h_k$ increase the variance and
decreases the bias of $(\nabla_{\mathrm{\scriptscriptstyle
FD}}v_{\mathrm{sim}})_k$ which is offset by the implicit averaging
in (\ref{KW}) and the increasing damping in (\ref{alpha_seq}).

The choice of $\gamma^{(0)}$, $B$, $\alpha_0$ and $h_k^{0}$ rests
upon us. Suppose the first three are fixed; for $h_k^0$, given
$\gamma^{(0)}=\gamma$, we suggest a value proportional to that which
minimizes the asymptotic (in $n$) mean square error of
(\ref{nabla_FD_IS}). This is
\begin{eqnarray}\label{h_opt} h_k^{*} &=&
\left(\frac{2 \sigma^2(\gamma)}{n v_{k}''(\gamma)^2}\right)^{1/3}
\end{eqnarray}where $n$ is the simulation size, and
$\sigma^2(\gamma)=\mathrm{Var}_{q}[(w\times
u)(\theta,\mathbf{y}_{\tau})]$.  This takes into account the use of
common random numbers for evaluating $(w\times
u)(\theta,\textbf{y}_{\tau_k})$ and $(w \times
u)(\theta,\textbf{y}_{\tau})$ in (\ref{nabla_FD_IS}). TODO: think
about the lack of smoothness of $\nabla_{\mathrm{FD}}
u_k(\theta,\mathbf{y}_{\tau})$ due to the fact that it is zero
outside $\mathbf{A}_k$.

We now explain how to estimate (\ref{h_opt}) from simulation data
with a bootstrap methodology. Start with an initial guess for
$h_k^*$, $h_k$, and update it as follows:
\begin{algorithmic}
\STATE $m\leftarrow 1$ \STATE $\bar\sigma^2\leftarrow 0$ \STATE
$\bar \upsilon''_k\leftarrow 0$ \REPEAT \STATE Generate
$s=\{(\theta,\mathbf{y}_{n})^{(j)}\}_{j=1}^n$ from
 $q$\STATE $\hat\sigma^2\leftarrow \widehat{\mathrm{Var}}_{q}[(w\times
u)(\theta,\mathbf{y}_{\tau})]$ \STATE $\bar\sigma^2\leftarrow
((m-1)\bar\sigma^2+\hat\sigma^2)/m$ \STATE Compute
$\hat\upsilon_{k}''$ by finite difference approximation with
increment $h_k$, based on $s$ \STATE $\bar \upsilon_{k}''\leftarrow
((m-1)\bar \upsilon_{k}''+\hat \upsilon_{k}'')/m$ \STATE
$h_k\leftarrow h_k(\bar\sigma^2,\bar \upsilon''_{k})$ by
(\ref{h_opt})\STATE $\hat\sigma_{ h_k}\leftarrow
\widehat{\mathrm{Var}}[h_k]$ \STATE $m\leftarrow m+1$ \UNTIL
$m>(\Phi^{-1}(1-\delta/2)\hat\sigma_{ h_k}/(| h_k|\epsilon))^2$
\end{algorithmic}The if statement ensures that the error for each $k$ is less than $\epsilon$,
with $1-\delta$ probability based on a normal approximation,
resulting in a formula, whose parameters $h_k^*$ and $\sigma_{h_k}$
are replaced by their estimate $h_k$ and $\hat\sigma_{h_k}$.

A byproduct of the above procedure is that we can reuse the estimate
of $\{v''(\gamma)\}_{k=1}^{2(n-1)}$ as the diagonal of $H$, provided
it is positive definite.
%\label{delta_opt_adaptive}
%P(|(\bar{\xi}_{m}-\xi)/\xi|>\epsilon)<\delta\Leftrightarrow
%m>(\Phi^{-1}(1-\delta/2)\sigma_{\hat\xi}/(|\xi|\epsilon))^2

\section{Method 2}\label{sec:meth:i}
Recall (\ref{upsilon_par_sr_i}). Without loss of generality we will
now ignore $\mathbf{c}_j$ and the upper bounds and focus on a given
$i$:
\begin{equation}
\upsilon(\boldsymbol{\gamma}_i)=\int_{\mathbb{R}}\idotsint\limits_{\mathbf{R}_{i-1}}\Bigl[\int_{-\infty}^{\gamma_i}u(i,\theta,T)p(\theta,\mathbf{y}_{i})
d y_i d \mathbf{y}_{i-1}  d \theta
\end{equation}
Let
$\upsilon_j(\boldsymbol{\gamma}_i)=\partial\upsilon(\boldsymbol{\gamma}_i)/\partial{\gamma_j}
$:
\begin{equation}
\upsilon_j(\boldsymbol{\gamma}_i)=\begin{cases}
0, & j>i \\
\int_{\mathbb{R}} \idotsint\limits_{\mathbf{R}_{i-1}}\Bigl[
u(i,\theta,T)p(\theta,\mathbf{y}_{i-1},\gamma_i) \Bigr]
d \mathbf{y}_{i-1}  d \theta , & j=i\\
\int_{\mathbb{R}} \idotsint\limits_{\mathbf{R}_{i-1}}\Bigl[
\int_{-\infty}^{\gamma_i}
u(i,\theta,T)p(\theta,\mathbf{y}_{j-1},\gamma_j,\mathbf{y}_{j<l\leq
i})\Bigr] d \mathbf{y}_{j-1}d \mathbf{y}_{j<l\leq i}  d \theta, &j<i
\end{cases}
\end{equation}
The third equality is similar to the second, so our focus will be on
the later:
\begin{align}
\upsilon_i(\boldsymbol{\gamma}_i)= \int_{\mathbb{R}}
\idotsint\limits_{\mathbf{R}_{i-1}}\Bigl[
u(i,\theta,T)p(\theta,\mathbf{y}_{i-1},\gamma_i) \Bigr] d
\mathbf{y}_{i-1}  d \theta
\end{align}Since the above quantity is bounded and $u(i,\theta,T)\geq 0$ we may
define the density
\begin{align}
L(\theta,\mathbf{y}_{i-1})&=\boldsymbol{1}\{\mathbf{y}_{i-1}\in
\mathbf{R}_i\}u(i,\theta,T)p(\theta,\mathbf{y}_{i-1},\gamma_i)/\upsilon_i(\boldsymbol{\gamma}_i)\\
&=\boldsymbol{1}\{\mathbf{y}_{i-1}\in \mathbf{R}_i,
\theta<\gamma_i\}\theta
p(\theta,\mathbf{y}_{i-1},\gamma_i)/\upsilon_i(\boldsymbol{\gamma}_i)
\end{align}
and call $L_{\mathrm{un}}(.)$ its un--normalized version.
Given
$q\gg p$, the importance sampling representation for this problem
is\begin{align}
\mathrm{E}_q[L_{\mathrm{un}}(\theta,\mathbf{y}_{i-1})/q(\theta,\mathbf{y}_{i-1})]
&=
\mathrm{E}_q[L(\theta,\mathbf{y}_{i-1})/q(\theta,\mathbf{y}_{i-1})]\upsilon_j(\boldsymbol{\gamma}_i)\\
&=\upsilon_j(\boldsymbol{\gamma}_i)
\end{align}or,
\begin{align}
\mathrm{E}_q[L_{\mathrm{un}}(\theta,\mathbf{y}_{i-1})/q_{\mathrm{un}}(\theta,\mathbf{y}_{i-1})]
&=\upsilon_j(\boldsymbol{\gamma}_i)/C_q\\
E_q[q_{\mathrm{un}}(\theta,\mathrm{y}_{i-1})]&=C_q
\end{align}Alternatively, we may determine a Markov chain $\{z_n, n\in
\mathbb{N}\}$ with initial distribution $K_0$ and transition
probabilities $\{K_n(.,.)\}_{n\in \mathbb{N}}$ which converges to
$L(.)$ in the limit as $n\rightarrow \infty$. Specifically,
$\eta_n(A)=\mathrm{Pr}(z_n\in A)$ is defined recursively as follows:
\begin{align}
\eta_0(A)\triangleq K_0(A)\\
\eta_n(A)\triangleq \mathrm{E}_{\eta_{n-1}}[K_n(z,A)]
\end{align}We want to find $K_0(.)$ and $\{K_n(.,.)\}_{n\in \mathbb{N}}$ such that
$\eta_{\infty}(A)=L(A)$. Given a choice of proposal initial and
transition distributions, $Q_0(.)$ and $Q_n(.)$, the
Metropolis--Hastings algorithm is appropriate, and is defined by
\begin{align}
\alpha(z_{n-1},z)&=\min\{\frac{Q_n(z_{n-1},z)L(z)}{Q_n(z,z_{n-1})L(z_{n-1})},1\}\\
K_n(z_{n-1},d{z_n}|z)&=\delta_{\{z\}}(z_n)\alpha(z_{n-1},z)+(1-\alpha(z_{n-1},z))\delta_{\{z_{n-1}\}}(z_n)\\
K_n(z_{n-1},d{z_n})&=E_{Q_n}[K_n(z_{n-1},d{z_n}|z)]
\end{align}The ratio formulation of $\alpha(.,.)$ ensures that it is
insensitive to replacing $L(.)$ $L_{\mathrm{un}}(.)$ or $Q_n(.)$ by
$Q_{n,\mathrm{un}}(.)$. Loosely speaking, we must ensure that
$(Q_0,\{Q_n\}_{n\in\mathbb{N}})$ will reach all regions $A$ such
that $\mathrm{Pr}(z \in A)>0$ from an arbitrary starting point,
within finite time. These conditions are usually easy to meet. Our
estimation procedure will be based on
\begin{equation}
E_{\eta_{\infty}}[1/L_{\mathrm{un}}(z)]=\upsilon_j(\boldsymbol{\gamma}_i)
\end{equation}
The difficulty in either of importance sampling or MCMC is to obtain
reasonable convergence rates. The challenge, here is 1) to find a
distribution that approximates $\theta
p(\theta,\mathbf{y}_{i-1},\gamma)\propto \theta
p(\theta,\mathbf{y}_{i-1})p(\gamma_1|\gamma)$ and 2) to sample from
the latter on the hypercube $\boldsymbol{1}\{\mathbf{y}_{i-1}\in
\mathbf{R}_i, \theta<\gamma_i\}$. In the case of an independence
chain for MCMC or importance sampling, please read the relevant
section in the appendix. In the MCMC case for random walk chain, an
adaptive procedure to maximize the jumping distance (weighted by the
probability of acceptance $\alpha(.,.)$) is presented in Pasarica
and Gelman, 2005).

So far, we have focused on $\upsilon_j(\boldsymbol{\gamma}_i)$ for a
particular $(i,j)$. We have to think about an efficient way to reuse
the sample across $(i,j)$'s.

The above discussion should apply to the extended model of the next
section (\ref{normal_normal}), after condition on $\sigma$.

%Each $\partial_{\gamma_j^{\pm}}\upsilon(\boldsymbol{\gamma}_i)$ is
%the form $\int_{\mathbb{R}}
%\idotsint\limits_{\mathbf{R}_{i-1}}\Bigl[
%u(i,\theta,d^{\pm})p(\theta,\mathbf{y}_{k-1},\gamma_k,\mathbf{y}_{k\leq
%l \leq i}) \Bigr] d \mathbf{y}_{i-1}  d \theta$.

\section{Model extensions}\label{sec:ext}
We begin by extending (\ref{normal_normal}), which relies on known
variances to the most immediate generalization:
\begin{align}\label{N_xi}
\theta|\sigma^2 &\sim N(y_0,\sigma^2/\kappa_0)\\
\sigma^2 &\sim \mathrm{Inv}-\chi(\nu_0,\sigma_0^2)
\end{align}The parameter $\kappa_0$ can be interpreted as the number of
prior measurements i.e. $\kappa_0+\tau$ is the total number of
measurements under a particular stopping rule.

Let us review standard properties:
\begin{equation}
p(\theta|\mathbf{y}_i)\sim
t_{\nu_i}(\theta|\overline{\mathbf{y}}_i,\sigma^2(\mathbf{y}_i)/\kappa_i)
\end{equation}where $\overline{\mathbf{y}}_i=(\kappa_0 y_0 +\sum_{j=1}^i
y_i)/(\kappa_0+i)$, $\kappa_i=\kappa_0+i$, $\nu_i=\nu_0+i$ and
$\nu_i\sigma^2(\mathbf{y}_i)=\nu_0\sigma_0^2+(i-1)s(\mathbf{y}_{0<j\leq
i})^2+\frac{\kappa_0(\kappa_0+i)}{i}(\overline{\mathbf{y}}_{i}-y_0)^2$
and $s^2(.)$ is the the sample variance.

%\section{Application}\label{sec:appl}
%Here we compare the performances of $p$, $q$, $q^*$ using steepest
%descent, with $n=40$ for the first, and $n=20$ for the last two (so
%that the three use the same \emph{total} number of simulation draws
%for each $b$). We set the ``stabilizer'' constant $B=10$ and adjust
%$\alpha_{0}$ via trial and error. The KW algorithm is run 20 times,
%with a different random seed for each; we summarize the results with
%univariate confidence intervals.

%The first three Figures are the initial parameters, sequence of
%arguments $(\gamma_{-}^{(b)},\gamma_{+}^{(b)})$, and sequence of
%Bayes risks, $-v(\gamma^{(b)})$, starting from a point where $v(.)$
%is ``flat'', and therefore difficult for KW. From figure 1, the
%optimal FD increments for $p$ are much greater (by up to an order of
%10) than those of $q$ and $q^*$, due to the greater variance of
%$v(.)$ of the former relative to the latter. The confidence
%intervals of $(\gamma_{-}^{(b)}, \gamma_{+}^{(b)})$ and
%$-v(\gamma^{(b)})$ are noticeably larger for $p$ than for $q$ and
%$q^*$. The last two, therefore, are more effective on average.
%However, for some KW runs, the basic algorithm performs better:
%because it is more erratic, it is also more likely to reach the
%region of interest \emph{faster}, we think.

%The last three Figures considers a region where $v(.)$ is better
%behaved and closer to the solution.

%\newpage
%\begin{figure}          %Figure1
%\centering
%\includegraphics[width=4in]{C:/Documents/work/Programs/Mathematica/projects/sequential/2006_06_12_test_carlin_tableform_SIMS1_FDINIT.eps}
%\caption{Confidence intervals for $h_{\mathrm{opt}}^{(0)}$,
%$\sigma^2(\gamma^{(0)})$, $\nabla v(\gamma^{(0)})$ and
%$v''(\gamma^{(0)})$ for each of the importance distributions $p$,
%$q^*$ and $q$at iteration $b=0$, for a prior mean of $0.021$
%(compared with a threshold, $\mu_1=-0.14$) where
%$(\gamma_{-}^{(0)},\gamma_{+}^{(0)})=(\mu_1-0.1,\mu_1+0.1)=(-0.244,-0.044)$}
%\end{figure}
%\begin{figure}          %Figure2
%\centering
%\includegraphics[width=4in]{C:/Documents/work/Programs/Mathematica/projects/sequential/2006_06_12_test_carlin_tableform_SIMS1_LOGOPT_LOGARGS.eps}
%\caption{Confidence intervals for
%$(\gamma_{-}^{(b)},\gamma_{+}^{(b)})$, $b=1,4,16$ starting at
%$(\gamma_{-}^{(0)},\gamma_{+}^{(0)})=(\mu_1-0.1,\mu_1+0.1)=(-0.244,-0.044)$,
%for each of $p$, $q^*$ and $q$at iteration $b=0$, for a prior mean
%of $0.021$ (compared with a threshold, $\mu_1=-0.14$)}
%\end{figure}
%\begin{figure}          %Figure3
%\centering
%\includegraphics[width=2.5in]{C:/Documents/work/Programs/Mathematica/projects/sequential/2006_06_12_test_carlin_tableform_SIMS1_LOGOPT_LOGVALS.eps}
%\caption{Confidence intervals for the Bayes risk (negative expected
%utility) for $b=1,4,16$ starting at
%$(\gamma_{-}^{(0)},\gamma_{+}^{(0)})=(\mu_1-0.1,\mu_1+0.1)=(-0.244,-0.044)$,
%for each of $p$, $q^*$ and $q$at iteration $b=0$, for a prior mean
%of $0.021$ (compared with a threshold, $\mu_1=-0.14$)}
%\end{figure}

%\begin{figure}          %Figure4
%\centering
%\includegraphics[width=4in]{C:/Documents/work/Programs/Mathematica/projects/sequential/2006_06_12_test_carlin_tableform_SIMS2_FDINIT.eps}
%\caption{Same as Figure 1, with
%$(\gamma_{-}^{(0)},\gamma_{+}^{(0)})=(\mu_1-0.1,\mu_1+0.1)=(-0.194,-0.044)$}
%\end{figure}

%\begin{figure}          %Figure5
%\centering
%\includegraphics[width=4in]{C:/Documents/work/Programs/Mathematica/projects/sequential/2006_06_12_test_carlin_tableform_SIMS2_LOGOPT_LOGARGS.eps}
%\caption{Same as Figure 2, with
%$(\gamma_{-}^{(0)},\gamma_{+}^{(0)})=(\mu_1-0.1,\mu_1+0.1)=(-0.194,-0.044)$}
%\end{figure}
%\begin{figure}          %Figure 6
%\centering
%\includegraphics[width=2.5in]{C:/Documents/work/Programs/Mathematica/projects/sequential/2006_06_12_test_carlin_tableform_SIMS2_LOGOPT_LOGVALS.eps}
%\caption{Same as Figure 3, with
%$(\gamma_{-}^{(0)},\gamma_{+}^{(0)})=(\mu_1-0.1,\mu_1+0.1)=(-0.194,-0.044)$}
% \end{figure}

\section{Appendix}
\subsection{Variance of importance sampling}

Here, we consider transformation of stopped paths such as
$(\theta,\mathbf{y}_\tau)\rightarrow (w \times u)
(\theta,\mathbf{y}_{\tau})$. Let $x_i \triangleq
\mu(\mathbf{y}_i)\}_{i=0}^{n}$. We recall from Section 2 that
conditional on $\theta$, $\{x_i\}_{i=1}^{n}$ evolves as a Markov
chain. Furthermore, remember that
$u(\theta,\mathbf{y}_{n})=u(\theta,\mathbf{y}_{\tau})$ because
$\tau>i\Rightarrow u(i,\theta,\mathbf{y}_{i})=0$. Therefore,
assuming Assume
$q(\theta,\mathbf{x}_i)=q(\theta)q(x_1|x_0)...q(x_{i}|x_{i-1})$,
\begin{align}
\upsilon&=\mathrm{E}_q[u(\theta,\mathbf{x}_{n})w(\theta,\mathbf{x}_{n})]\\
&=\mathrm{E}_q[E_q[u(\theta,\mathbf{x}_{\tau})w(\theta,\mathbf{x}_{\tau})\mathrm{E}_q[w(\theta,\mathbf{x}_{n})/w(\theta,\mathbf{x}_{\tau})|\theta,\mathbf{x}_{\tau}]|\theta]]\\
&=\mathrm{E}_q[u(\theta,\mathbf{x}_{\tau})w(\theta,\mathbf{x}_{\tau})]
\end{align}Is this still valid with $\tilde w$, the normalized
weight?

The advantage of the first equality is that we can work with
$z=(\theta,\mathbf{y}_{n})$ which is a vector variable of fixed
size. The formula for this variance using normalized weights is:
\begin{align}
\mathrm{Var}_q[\tilde\upsilon]&=\mathrm{Var}_q[\sum_{j=1}^n
(\tilde w \times u)(z_j)]\\
&\rightarrow\frac{1}{n}\mathrm{E}_p[\tilde w(z)(u(z)-\upsilon)^2]
\end{align}

\subsection{Bias variance decomposition of the finite difference method}
To begin, let's assume that $q(.)=p(.)$, $n=2$ and consider the
univariate case by fixing $k=1$. Fix $h>0$, so that
$\tau(\gamma+h)\leq\tau(\gamma)$ and let $u(\gamma)\triangleq
u(\theta,\mathbf{y}_{\tau(\gamma)})$, so that
$\upsilon(\gamma)=E[u(\gamma)]$:
\begin{align}
\nabla_\mathrm{\scriptscriptstyle FD}^h
u(\gamma)&=\frac{u(\gamma+h)-
u(\gamma)}{h}\\&=\label{events}\begin{cases}
0&, \tau(\gamma)=\tau(\gamma+h)\\
u(1,\theta,P)-u(2,\theta,P)&, \tau(\gamma)>\tau(\gamma+h) \quad \mathrm{and}\quad a(\mathbf{y}_\tau)=P\\
u(2,\theta,P)-u(2,\theta,T)&, \tau(\gamma)>\tau(\gamma+h) \quad
\mathrm{and}\quad a(\mathbf{y}_\tau)=T
\end{cases}.
\end{align}
\begin{align}\textsl{bias(h)}&\triangleq
\mathrm{E}[\nabla_\mathrm{\scriptscriptstyle FD}^h u(\gamma)-\nabla
\upsilon(\gamma)]\\&=\nabla_\mathrm{\scriptscriptstyle FD}^h
\upsilon(\gamma)-\nabla \upsilon(\gamma)
\end{align}
Provided $v(\gamma+h)=v(\gamma)+\nabla v(\gamma)h+\frac{1}{2}
\nabla_2 v(\gamma)h^2+o(h^2)$,
\begin{align}
\textsl{bias}(h)&=\frac{1}{2} \nabla_2 v(\gamma)h+o(h)\\&=O(|h|)
\end{align}We need to show that
\begin{align}
\textsl{var}(h)&\triangleq\mathrm{Var}[\nabla_\mathrm{\scriptscriptstyle
FD}^h u(\gamma)]\\&=O(1/|h|)
\end{align}This is not guaranteed, given the discrete nature of
(\ref{events}) (TODO).

We may rewrite the order of the bias and variance, for an iid sample
of size $n$ as
\begin{align}
\textsl{bias}(h)& = b h+o(|h|)\\
\textsl{var}(h)& = \frac{1}{n h}\sigma^2 +o(1/|h|)
\end{align}with $b=\nabla_2 v(\gamma)$. Suppose $h=h_* n^{-\eta}$
and plug that expression into the above system. Then
\begin{align}
\textsl{mse}(h,n)&=\textsl{bias}^2(n,h)+\textsl{var}^2(n,h)\\
&=b^2 (h_* n)^{-2\eta}+\sigma^2 h_*^{\eta}n^{\eta-1}
\end{align}This is minimized for $\eta=1/3$. By plugging that value
in the above, the new expression is minimized for
\begin{align}
h_*=\left(\frac{2\sigma^2}{(\upsilon'')^2}\right)^{1/3}
\end{align}This results derive from Glynn, 1989.

\subsection{Stratification}
Consider $\mathbf{x}^{\mathrm{T}}\sim N(\boldsymbol{\mu},\Sigma)$
where $\mathbf{x}$ has dimension $n$. Consider the linear
restrictions:
\begin{align}
\mathbf{R}=\{\mathbf{x}:\gamma_i^{-}\leq x_i \leq
\gamma_{i}^{+},\quad i\in \}
\end{align}where the $\gamma_i^{-}=\infty$ and/or
$\gamma_i^{\pm}=\pm\infty$ are allowed for some $i$, but for at
least one $i$, $|\gamma_i^{\pm}|<\infty$. Our problem is to sample
from $p_*(\mathbf{x})\propto
p(\mathbf{x})\boldsymbol{1}_{[\mathbf{R}]}(\mathbf{x})$.

Let's consider the unrestricted case first. A standard property of
the normal distribution is:
\begin{equation}
x_i|\mathbf{x}_{-i}\sim N(\mu_{1|-1}|\Sigma_{1|-1})
\end{equation}where
\begin{align}
\mu_{1|-1}&=\mu_i+\Sigma_{i,- i}\Sigma_{-i,-i}^{-1}(\mathbf{x}_{-
i}-\boldsymbol{\mu}_{-i})\\ \Sigma_{1|-1}&=\Sigma_{ii}-\Sigma_{i,-
i}\Sigma_{-i,-i}^{-1}\Sigma_{-i,i}
\end{align}
By sampling $x_i\sim $, and iterating over all $i$'s, we obtain a
sample from the desired multivariate distribution. This is a
particular application the Gibbs algorithm. It's efficiency is
inversely related to $n$ and the degree of positive correlation
between the $x_i$'s. However, it is of interest to us as it is
relatively straightforward to sample from a truncated univariate
normal, by the method of inverse cdf:
\begin{algorithmic}
\STATE $\gamma_i^{\pm}\leftarrow
(\gamma_i^{\pm}-\mu_{i|-i})/\Sigma_{i|-i}$ \STATE
$u\sim\mathrm{Unif}[\Phi(\gamma_i^{-}),\Phi(\gamma_i^{+})]$ \STATE $
x_i^*\leftarrow \Sigma_{i|-i}\Phi^{-1}(u)+\mu_{i|-i}$
\end{algorithmic}

For the reasons previously mentioned, we need to consider a more
efficient algorithm. Philippe and Robert, 2003 propose a perfect
sampling scheme. Troughton and Godsill, propose a ``Gaussian
windowing'' approach with intuitive appeal and simplicity. Consider
as candidate density, $q(.)$
\begin{align}
N(\mathbf{x}|\boldsymbol{\mu}_*,\Sigma_*)\propto
N(\mathbf{x}|\boldsymbol{\mu},\Sigma)N(\mathbf{x}|\boldsymbol{\bar\gamma},\times
 \Lambda_{\kappa})
\end{align}where
\begin{align}
\boldsymbol{\bar\gamma}
&=(\boldsymbol{\gamma}^{-}+\boldsymbol{\gamma}^{-})/2\\
\Delta\boldsymbol\gamma&=\boldsymbol{\gamma}^{+}-\boldsymbol{\gamma}^{-}\\
\Lambda_{\kappa}&=\kappa\times\textsl{diag}(\Delta\boldsymbol\gamma)
\end{align}so that
\begin{align}
\Sigma_*&=(\Sigma^{-1}+\Lambda_{\kappa}^{-1})^{-1}\\
\mu_*&=\Sigma_*(\Sigma^{-1}\boldsymbol{\mu}+\Lambda_{\kappa}^{-1}\boldsymbol{\bar{\gamma}})
\end{align}We are no longer restricted to sample from each
conditional univariate distribution. Furthermore, the mean is a
weighted average of the true mean and the center of the bounds.
Decreasing $\kappa$ increases the probability that the draws will
fall in the hypercube; conditional on that event, however,
$w(.)=q(.)/p(.)$ will be more variable. In this context, the
adaptive importance sampling method described below seems highly
relevant, likewise for MCMC.

\subsection{Adaptive importance sampling}
We recall that $\mu$ is the prior mean in $p(.)$. Here, we define a
family of sampling distributions indexed by $\lambda$:
$\{q_{\lambda}(.),\lambda\in \Lambda\}$. In the case of model
(\ref{normal_normal}), this will be, for example,
$q_{\lambda}(.)=p(.|\mu=\lambda)$.

For a vector $z\sim q_*(z)$, and a quantity of interest
$g=\mathrm{E}[z]$, the method traditionally proceeds as follows (see
Oh and Berger, 1992). First we find $\Lambda$ such that
$\mathrm{E}_\lambda[\Lambda(z)]=\lambda$. Starting from and initial
guess $\lambda=\lambda_0$, we update it as follows
\begin{algorithmic}
\STATE $j\leftarrow 0$; $n_0\leftarrow 0$ \FOR{$j=1$ to $J$} \STATE
Generate an iid sample $z_1,...,z_{n_{j}}$ from $q_{\lambda_{j-1}}$
\STATE
$\lambda_{j}\leftarrow\hat{\mathrm{E}}_{\lambda_{j-1}}[w_{j-1}(z)\Lambda(z)]$
\STATE $\hat g \leftarrow (N_{j-1} \hat g + n_j
\hat{\mathrm{E}}_{\lambda_1}[w_1(z)z])/(N_{j-1}+n_j)$ \STATE
$N_{j-1}\leftarrow N_{j-1}+n_j$\ENDFOR
\end{algorithmic}
We would expect $\lambda
\xrightarrow{p}\lambda_*=\mathrm{E}_*[\Lambda(z)]$ and therefore
this method will be most effective if $q_{\lambda_*}$ is a good
approximation to $q_*(.)$. The dimensionality of $\Lambda$ should be
inversely related to expected variability of the data.

A possible variant is to update $\lambda$ the same way we update
$g$, that is by smoothing it across iterations. Moreover the
coefficient of smoothing can be set according to the estimated
variability in each sample. Finally, we can replace the moment
matching step by a sample variance minimizing step, that is
$\lambda_l = \min_\lambda
\widehat{\mathrm{Var}}_{\lambda_{l-1}}[w_{l-1}(z) z]$ (see Pasarica
and Gelman, 2005 for a successful implementation of this idea in a
Markov chain context).

%\subsection{Some Algebra}
%The \textit{Woodbury} formula (see Lange, 1996):
%\begin{align}\label{SMW}
%(A+B C D)^{-1}=A^{-1}[I-B(C^{-1}+D A^{\mathrm{-1}}B)^{-1}D A^{-1}]
%\end{align}
%Consider the decomposition of a positive definite matrix
%$\Sigma$:\begin{align}\Sigma=\Gamma \Lambda_{\Sigma}
%\Gamma^{\mathrm{T}}\end{align} where $\Gamma.\Gamma^{\mathrm{T}}=I$
%and $\Lambda_{\Sigma}$ is a diagonal matrix. This implies
%$\Sigma^{-1}=\Gamma \Lambda_{\Sigma}^{-1} \Gamma^{T}$. With
%$B\triangleq \Gamma$, $D\triangleq \Gamma^{T}$, $C\triangleq
%\Lambda_{\Sigma}^{-1}$, and $A=\Lambda^{-1}$ another diagonal
%matrix, we get by (\ref{SMW}) is
%\begin{align}\label{SMW}
%(\Sigma^{-1}+\Lambda^{-1})^{-1}&=\Lambda[I-\Gamma(\Lambda_{\Sigma}+\Gamma^{\mathrm{T}}
%\Lambda \Gamma)^{-1}\Gamma^{T} \Lambda]
%\end{align}

\subsection{misc}
Given $x\sim N(\mu,\sigma^2)$,
\begin{align}
\mathrm{E}[x 1\{a\leq x\leq
  b\}] = &-\sigma(\phi((b-\mu)/\sigma)\\\nonumber
  & -\phi((a-\mu)/\sigma)\\\nonumber
  & +\mu(\Phi((b-\mu)/\sigma)\\\nonumber
    & -\Phi((a-\mu)\sigma))
\end{align}

\section*{Bibliography}
\printbibliography[heading=none]

\end{document}
