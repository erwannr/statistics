\documentclass[11pt]{article}
\usepackage[
bibencoding=auto
,backend=biber
,sorting=ynt
% ,style=verbose-ibid
% ,citetracker
% ,sorting=none
% ,autolang%error: no value specified for autolang.
]{biblatex}
\usepackage{csquotes, datetime2, keyfloat}
%\newcounter{res}
%\newtheorem{result}[res]{Result}
%\usepackage{filecontents}
\addbibresource{../rogard.bib}% syntax for version >= 1.2
\usepackage{amsmath,amssymb,amsthm}
\usepackage{datetime2, csquotes}
%---
\newcommand{\tangnt}{t^{\mathrm{u}}}
\newcommand{\linejoin}{t^{\mathrm{l}}}
\newcommand{\upperlog}{u}
\newcommand{\lowerlog}{l}
\newcommand{\proposal}{s}
\newcommand{\normconst}{c}
\newcommand{\cumulproposal}{S}

\newcommand{\arsindex}{k}
\newcommand{\arsmaxindex}{K}
% ---
% ---
\ProvideDocumentCommand{\origDate}{}{\DTMdate{2006-11-08}}
\ProvideDocumentCommand{\ThisDate}{}{\DTMdate{2022-04-06}}
% ---

\title{Derivative Based Adaptive Rejection Sampling\\ (a review)}
\author{Erwann Rogard}
\date{\ThisDate}

\usepackage{hyperref}
\begin{document}

\maketitle

% This algorithm generates draws from an arbitrary log-concave pdf using only the log pdf (up to an additive constant) and the derivative. This algorithm is commonly used for sampling from full conditional distributions arising in Gibbs sampling. 

\begin{abstract}This review was first published on \DTMdate{2007-11-19} as companion to a Mathematica package\footnote{\url{https://library.wolfram.com/infocenter/ID/7071/}}. The present version is just a revamp\footnote{Revisions:\url{https://github.com/erwannr/statistics/commits/main/deriv_ars}}.
\end{abstract}

The purpose of the algorithm is to sample from a univariate log--concave (target) density, known up to a constant of proportionality, as commonly arising in Gibbs sampling applications\cite{gilks1992}\cite{wild1993}. The algorithm is derived from rejection sampling but instead of being pre--specified, the covering (proposal) density is constructed adaptively using only evaluations of the un--normalized density and its derivative. Geometrically, the log of the former is the upper envelope of the convex hull delimited by the tangents to the log un--normalized target at the evaluation points. This modification to the original algorithm requires a new rejection rule which depends on the distances between the true density and the upper hull, evaluated at the candidate draw, and that of the latter to a lower hull, formed by joining the evaluation points. Consider the definitions:
\begin{center}
\begin{tabular}{|@{\hspace{1ex}}l@{\hspace{1ex}}|@{\hspace{1ex}}l@{\hspace{1ex}}|@{\hspace{1ex}}p{2.9in}@{\hspace{1ex}}|}\hline
\multicolumn{3}{|@{\hspace{1ex}}l|}{Definitions}\\\hline
$D$ & & domain (connected)\\
$D_{\pm}$ & & lower/upper bound\\
$f(.)$ & & log--concave density\\
$g(.)$ & $f(.) \propto$ g(.) & un--normalized density\\
$h(.)$ & $\log g(.)$& \\
$h'(.)$ & $h'(x)=\partial h(.)/\partial x$&\\\hline
$x_{\arsindex}$ &  $x_{\arsindex}\in D$& $\arsindex$th abscicae\\
$x_{(\arsindex)}$ & $x_{(\arsindex)}\leq x_{(\arsindex+1)}$ & $\arsindex$th ordered abscicae\\
$\tangnt_{\arsindex}(.)$ &  & tangent to $h(.)$ at $x_{(\arsindex)}$, $\arsindex=1,...\arsmaxindex$\\
$\linejoin_{\arsindex}(.)$ &  & line through $\{(x_{(j)},h(x_{(j)})):j=\arsindex,\arsindex+1\}$, $\arsindex=1,...,\arsmaxindex-1$\\
$X$ & $\{x_\arsindex:\arsindex=1,...,\arsmaxindex\}$  &  set of $\arsmaxindex\geq 2$ points such that $h'(x_{(1)})>0$ and $h'(x_{(\arsmaxindex)})<0$, if $D_{-}=-\infty$ and , $D_{+}=+\infty$, respectively\\
$z_{(\arsindex)}$ & $\tangnt_{\arsindex}(z_{(\arsindex)})=\tangnt_{\arsindex+1}(z_{(\arsindex)})$ & tangent intersection abscicae, $\arsindex=1,...,\arsmaxindex-1$\\
$z_{(0)}$ & $D_{-}$ & \\
$z_{(\arsmaxindex)}$ & $D_{+}$ & \\
$\upperlog(.)$ &  & upper hull envelope of $h(.)$\\
$\proposal(.)$ & $\proposal(x)\propto \exp(\upperlog(x))$ & proposal density \\
$\normconst$ & $\int_D \exp(\upperlog(x)) dx$ & normalizing constant \\
$\cumulproposal(.)$ & $\int_{z_{(0)}}^{(.)} s(x) dx$ & cumulative density\\
$\lowerlog(.)$ & & lower hull envelope \\\hline
\end{tabular}
\end{center}

In particular, 
\begin{align}
\tangnt_\arsindex(x) &=h(x_{(\arsindex)})+(x-x_{(\arsindex)})h'(x_{(\arsindex)})\\
\linejoin_\arsindex(x) &=h(x_{(\arsindex)})+(x-x_{(\arsindex)})\frac{h(x_{(\arsindex+1)})-h(x_{(\arsindex)})}{x_{(\arsindex+1)}-x_{(\arsindex)}}\\
z_{(\arsindex)}&=-(\tangnt_{\arsindex+1}(0)-\tangnt_{\arsindex}(0))/(h'(x_{(\arsindex+1)})-h'(x_{(\arsindex)}))\\
\upperlog(x)&=\sum_{\arsindex=1}^\arsmaxindex 1{\{z_{(\arsindex-1)}\leq x <z_{(\arsindex)}\}} \tangnt_{\arsindex}(x)\\
\cumulproposal^\upperlog_{\mathrm{un}}(x)&= 
\sum_{\arsindex=1}^{\arsmaxindex}\int_{x\wedge z_{(\arsindex-1)}}^{x\wedge z_{(\arsindex)}}\exp(\tangnt_\arsindex(y))dy\\
&=\sum_{\arsindex=1}^{\arsmaxindex}\frac{1}{h'(x_{(\arsindex)})}\left( \exp(\tangnt_\arsindex(x\wedge z_{(\arsindex)}))-\exp(\tangnt_\arsindex(x\wedge z_{(\arsindex-1)}))\right)\\
\cumulproposal_\upperlog^{-1}(u)&=\inf\{x:\cumulproposal^{\mathrm{un}}_\upperlog(x)\geq u \normconst\}\\\label{eq:invcumul_1}
&=x_{(\arsindex_*)}+\frac{1}{h'(x_{(\arsindex_*)})}\left(\log\left(e^{\tangnt_{\arsindex_*}(z_{(\arsindex_*-1)})}+h'(x_{(\arsindex_*)})(\normconst u-\cumulproposal^{\mathrm{un}}_\upperlog(z_{(\arsindex_*-1)}))\right)-h(x_{(\arsindex_*)})\right)\\\label{eq:invcumul_2}
%&=z_{(i_*-1)}+\frac{1}{h'(x_{(i_*)})}\log\left(1+e^{-\tangnt_{i_*}(z_{(i_*-1)})}h'(x_{(i_*)})\normconst(u-\cumulproposal_{\mathrm{un}}^\upperlog(z_{(i_*-1)}))\right)\\
%\normconst(u-\cumulproposal_{\mathrm{un}}^\upperlog(z_{(i_*-1)}))}{\exp(\tangnt(z_{(i_*-1)}))}\right)\\
\lowerlog(x)&=\sum_{\arsindex=1}^{\arsmaxindex-1}1\{x_{(\arsindex)}\leq x<x_{(\arsindex+1)}\}\linejoin_{\arsindex}(x)-\infty(1-1\{x_{(1)}\leq x<x_{(\arsmaxindex)}\})
\end{align}where $\arsindex_*=\min\{\arsindex:\cumulproposal^\upperlog(z_{(\arsindex)})\geq u\}$. %The presence of of the exponential term in (\ref{eq:invcumul_1}) may generate overflows and underflows, for the equivalent representations (\ref{eq:invcumul_2}), respectively.

The exponential terms in the un--normalized cumulative density are suceptible to overflows. To avoid this we do
\begin{align}
 h(.)\leftarrow h(.) - \max_{\arsindex}h(z_{(\arsindex)})
\end{align}which is permissible because $h(.)$ is the log of an un--normalized version of $f(.)$.

The algorithm below, initialized with some suitable $X$, generates one draw from $f$:
\theoremstyle{remark}
\newtheorem{algorithm}{Algorithm}
\begin{algorithm}[ARS]\label{alg:ARS}
\hfill\par
\begin{enumerate}
\item $\omega_l \stackrel{\mathrm{iid}}{\sim}\mathrm{Unif}(0,1),l\in\{0,1,2\}$; $x^*\leftarrow \cumulproposal_\upperlog^{-1}(\omega_0)$
\item if($\omega_1\leq \exp(l(x^*)-u(x^*))$)\{ return $x^*$ \}else\{\\
 if($\omega_2\leq \exp(h(x^*)-u(x^*)$)
\{ \begin{enumerate}
\item $X\leftarrow (X,x^*)$, $\arsmaxindex\leftarrow \arsmaxindex+1$, \item update derived quantities; \item return $x^*$
\end{enumerate}\}
else\{ goto 1.\}\\
\}
\end{enumerate}
\end{algorithm}

%A function object is a more general concept than a function because a function object can have state that persist across several calls (like a static local variable) and can be initialized and examined from outside the object (unlike a static local variable). For example:

\section*{Bibliography}
\printbibliography[heading=none]

\end{document}
